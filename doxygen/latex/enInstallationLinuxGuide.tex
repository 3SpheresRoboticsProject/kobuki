\section{\-Overview}\label{koMainPage_Overview}
\-This section describes the installation process for \-Linux.\subsection{\-Target audience}\label{enMainPage_who}
\-Linux people who want to compile \-Kobuki driver, but not use ros\section{\-Procedure}\label{enInstallationLinuxGuide_Procedure}
\subsection{\-Prerequirements}\label{enInstallationLinuxGuide_prereq}
\-We use catkin to build. \-First of all you need to install some basic dependencies\-:


\begin{DoxyItemize}
\item {\itshape cmake\/} -\/ \-A cross-\/platform, open-\/source build system.
\item {\itshape python\/} -\/ \-Python is a general-\/purpose, interpreted high-\/level programming language.
\item {\itshape wstool\/} -\/ \-Workspace tool for downloading sources.
\item {\itshape catkin\-\_\-pkg\/} -\/ \-A \-Python runtime library for catkin.
\item {\itshape empy\/} -\/ \-A \-Python template library.
\item {\itshape nose\/} -\/ \-A \-Python testing framework.
\end{DoxyItemize}


\begin{DoxyCode}
  sudo apt-get install python-wstool cmake python-catkin-pkg python-empy python
      -nose python-setuptools build-essential
\end{DoxyCode}


\-If you are {\bfseries not} on \-Ubuntu you can typically find cmake, python in your system repos and at worst, you can install python modules from {\tt \-Py\-Pi} via pip.

\-Refer to the {\tt \-Catkin documentation} for more details.\subsection{\-Catkin workspace}\label{enInstallationLinuxGuide_catkin}
\-Next we must prepare a catkin workspace.


\begin{DoxyCode}
  mkdir ~/kobuki_core
  wstool init -j5 ~/kobuki_core/src https://
      raw.github.com/yujinrobot/kobuki_core/hydro/kobuki_core.rosinstall
  cd ~/kobuki_core
  export PATH=~/tmp/kobuki_core/src/catkin/bin/:${PATH}
  catkin_make
  cd build; make install
\end{DoxyCode}


\-The utility catkin\-\_\-make is just a wrapper around cmake/make. \-You can just as easily call cmake ../src with the appropriate variables for \-C\-M\-A\-K\-E\-\_\-\-I\-N\-S\-T\-A\-L\-L\-\_\-\-P\-R\-E\-F\-I\-X amongst others and proceed from there. catkin\-\_\-make has a few options to allow you to indirectly set these cmake variables (again, \-C\-M\-A\-K\-E\-\_\-\-I\-N\-S\-T\-A\-L\-L\-\_\-\-P\-R\-E\-F\-I\-X is a useful one, otherwise it defaults to ./install).\subsection{\-Testing your installation}\label{enInstallationLinuxGuide_test}
\-You can test your installation by executing some of the demo and test programs provided with \-Kobuki driver. \-You must point your {\itshape \-L\-D\-\_\-\-L\-I\-B\-R\-A\-R\-Y\-\_\-\-P\-A\-T\-H\/} variable to the installed libraries. \-For example, if you type the following commands, the robot should move making squares\-:


\begin{DoxyCode}
  export LD_LIBRARY_PATH=~/kobuki_core/install/lib
  ~/kobuki_core/install/lib/kobuki_driver/demo_kobuki_simple_loop
\end{DoxyCode}
\section{\-Cross compiling}\label{enInstallationLinuxGuide_crossc}
\-We still have not prepared proper cross-\/compiling documentation, but have done so and we will be happy to help you.\section{\-Additional support}\label{enInstallationLinuxGuide_support}
\-Do not hesitate to ask on {\tt \-Kobuki users list} if you need additional support. 